%% Generated by Sphinx.
\def\sphinxdocclass{report}
\documentclass[letterpaper,10pt,english]{sphinxmanual}
\ifdefined\pdfpxdimen
   \let\sphinxpxdimen\pdfpxdimen\else\newdimen\sphinxpxdimen
\fi \sphinxpxdimen=.75bp\relax
\ifdefined\pdfimageresolution
    \pdfimageresolution= \numexpr \dimexpr1in\relax/\sphinxpxdimen\relax
\fi
%% let collapsible pdf bookmarks panel have high depth per default
\PassOptionsToPackage{bookmarksdepth=5}{hyperref}

\PassOptionsToPackage{booktabs}{sphinx}
\PassOptionsToPackage{colorrows}{sphinx}

\PassOptionsToPackage{warn}{textcomp}
\usepackage[utf8]{inputenc}
\ifdefined\DeclareUnicodeCharacter
% support both utf8 and utf8x syntaxes
  \ifdefined\DeclareUnicodeCharacterAsOptional
    \def\sphinxDUC#1{\DeclareUnicodeCharacter{"#1}}
  \else
    \let\sphinxDUC\DeclareUnicodeCharacter
  \fi
  \sphinxDUC{00A0}{\nobreakspace}
  \sphinxDUC{2500}{\sphinxunichar{2500}}
  \sphinxDUC{2502}{\sphinxunichar{2502}}
  \sphinxDUC{2514}{\sphinxunichar{2514}}
  \sphinxDUC{251C}{\sphinxunichar{251C}}
  \sphinxDUC{2572}{\textbackslash}
\fi
\usepackage{cmap}
\usepackage[T1]{fontenc}
\usepackage{amsmath,amssymb,amstext}
\usepackage{babel}



\usepackage{tgtermes}
\usepackage{tgheros}
\renewcommand{\ttdefault}{txtt}



\usepackage[Bjarne]{fncychap}
\usepackage{sphinx}

\fvset{fontsize=auto}
\usepackage{geometry}


% Include hyperref last.
\usepackage{hyperref}
% Fix anchor placement for figures with captions.
\usepackage{hypcap}% it must be loaded after hyperref.
% Set up styles of URL: it should be placed after hyperref.
\urlstyle{same}

\addto\captionsenglish{\renewcommand{\contentsname}{Contents:}}

\usepackage{sphinxmessages}
\setcounter{tocdepth}{1}



\title{MARIGOLD}
\date{Apr 05, 2024}
\release{0.0.1}
\author{adix}
\newcommand{\sphinxlogo}{\vbox{}}
\renewcommand{\releasename}{Release}
\makeindex
\begin{document}

\ifdefined\shorthandoff
  \ifnum\catcode`\=\string=\active\shorthandoff{=}\fi
  \ifnum\catcode`\"=\active\shorthandoff{"}\fi
\fi

\pagestyle{empty}
\sphinxmaketitle
\pagestyle{plain}
\sphinxtableofcontents
\pagestyle{normal}
\phantomsection\label{\detokenize{index::doc}}


\sphinxAtStartPar
{\color{red}\bfseries{}*}M*ultiphase {\color{red}\bfseries{}*}A*nalysis of {\color{red}\bfseries{}*}R*aw {\color{red}\bfseries{}*}I*nformation for {\color{red}\bfseries{}*}G*lobal {\color{red}\bfseries{}*}O*r {\color{red}\bfseries{}*}L*ocal {\color{red}\bfseries{}*}D*ata


\begin{savenotes}\sphinxattablestart
\sphinxthistablewithglobalstyle
\sphinxthistablewithnovlinesstyle
\centering
\begin{tabulary}{\linewidth}[t]{\X{1}{2}\X{1}{2}}
\sphinxtoprule
\sphinxtableatstartofbodyhook
\sphinxAtStartPar
{\hyperref[\detokenize{modules/MARIGOLD.Condition:MARIGOLD.Condition}]{\sphinxcrossref{\sphinxcode{\sphinxupquote{MARIGOLD.Condition}}}}}(jgref, jgloc, jf, theta, ...)
&
\sphinxAtStartPar
Class to handle the local probe data
\\
\sphinxhline
\sphinxAtStartPar
{\hyperref[\detokenize{modules/MARIGOLD.extracts_and_loads:module-MARIGOLD.extracts_and_loads}]{\sphinxcrossref{\sphinxcode{\sphinxupquote{MARIGOLD.extracts\_and\_loads}}}}}
&
\sphinxAtStartPar

\\
\sphinxbottomrule
\end{tabulary}
\sphinxtableafterendhook\par
\sphinxattableend\end{savenotes}

\sphinxstepscope


\chapter{MARIGOLD.Condition}
\label{\detokenize{modules/MARIGOLD.Condition:marigold-condition}}\label{\detokenize{modules/MARIGOLD.Condition::doc}}\index{Condition (class in MARIGOLD)@\spxentry{Condition}\spxextra{class in MARIGOLD}}

\begin{fulllineitems}
\phantomsection\label{\detokenize{modules/MARIGOLD.Condition:MARIGOLD.Condition}}
\pysigstartsignatures
\pysiglinewithargsret{\sphinxbfcode{\sphinxupquote{class\DUrole{w}{ }}}\sphinxcode{\sphinxupquote{MARIGOLD.}}\sphinxbfcode{\sphinxupquote{Condition}}}{\sphinxparam{\DUrole{n}{jgref}\DUrole{p}{:}\DUrole{w}{ }\DUrole{n}{float}}\sphinxparamcomma \sphinxparam{\DUrole{n}{jgloc}\DUrole{p}{:}\DUrole{w}{ }\DUrole{n}{float}}\sphinxparamcomma \sphinxparam{\DUrole{n}{jf}\DUrole{p}{:}\DUrole{w}{ }\DUrole{n}{float}}\sphinxparamcomma \sphinxparam{\DUrole{n}{theta}\DUrole{p}{:}\DUrole{w}{ }\DUrole{n}{int}}\sphinxparamcomma \sphinxparam{\DUrole{n}{port}\DUrole{p}{:}\DUrole{w}{ }\DUrole{n}{str}}\sphinxparamcomma \sphinxparam{\DUrole{n}{database}\DUrole{p}{:}\DUrole{w}{ }\DUrole{n}{str}}}{}
\pysigstopsignatures
\sphinxAtStartPar
Class to handle the local probe data

\sphinxAtStartPar
Data is stored in the Condition.phi property. It’s actually 3 layers of dictionary
phi {[}angle{]} gives a dictionary with the various r/R
phi {[}angle{]}{[}r/R{]} gives a dictionary with the MIDAS output
The MIDAS output is itself a dictionary, with the keys listed in the “tab\_keys” array
So phi{[}angle{]}{[}r/R{]}{[}‘alpha’{]} should give you the void fraction at r/R for phi = angle
This structure is initialized with zeros for the MIDAS output at the pipe center and wall
\index{\_\_init\_\_() (MARIGOLD.Condition method)@\spxentry{\_\_init\_\_()}\spxextra{MARIGOLD.Condition method}}

\begin{fulllineitems}
\phantomsection\label{\detokenize{modules/MARIGOLD.Condition:MARIGOLD.Condition.__init__}}
\pysigstartsignatures
\pysiglinewithargsret{\sphinxbfcode{\sphinxupquote{\_\_init\_\_}}}{\sphinxparam{\DUrole{n}{jgref}\DUrole{p}{:}\DUrole{w}{ }\DUrole{n}{float}}\sphinxparamcomma \sphinxparam{\DUrole{n}{jgloc}\DUrole{p}{:}\DUrole{w}{ }\DUrole{n}{float}}\sphinxparamcomma \sphinxparam{\DUrole{n}{jf}\DUrole{p}{:}\DUrole{w}{ }\DUrole{n}{float}}\sphinxparamcomma \sphinxparam{\DUrole{n}{theta}\DUrole{p}{:}\DUrole{w}{ }\DUrole{n}{int}}\sphinxparamcomma \sphinxparam{\DUrole{n}{port}\DUrole{p}{:}\DUrole{w}{ }\DUrole{n}{str}}\sphinxparamcomma \sphinxparam{\DUrole{n}{database}\DUrole{p}{:}\DUrole{w}{ }\DUrole{n}{str}}}{{ $\rightarrow$ None}}
\pysigstopsignatures
\end{fulllineitems}

\subsubsection*{Methods}


\begin{savenotes}
\sphinxatlongtablestart
\sphinxthistablewithglobalstyle
\sphinxthistablewithnovlinesstyle
\makeatletter
  \LTleft \@totalleftmargin plus1fill
  \LTright\dimexpr\columnwidth-\@totalleftmargin-\linewidth\relax plus1fill
\makeatother
\begin{longtable}{\X{1}{2}\X{1}{2}}
\sphinxtoprule
\endfirsthead

\multicolumn{2}{c}{\sphinxnorowcolor
    \makebox[0pt]{\sphinxtablecontinued{\tablename\ \thetable{} \textendash{} continued from previous page}}%
}\\
\sphinxtoprule
\endhead

\sphinxbottomrule
\multicolumn{2}{r}{\sphinxnorowcolor
    \makebox[0pt][r]{\sphinxtablecontinued{continues on next page}}%
}\\
\endfoot

\endlastfoot
\sphinxtableatstartofbodyhook

\sphinxAtStartPar
\sphinxcode{\sphinxupquote{TD\_FR\_ID}}()
&
\sphinxAtStartPar

\\
\sphinxhline
\sphinxAtStartPar
\sphinxcode{\sphinxupquote{\_\_init\_\_}}(jgref, jgloc, jf, theta, port, database)
&
\sphinxAtStartPar

\\
\sphinxhline
\sphinxAtStartPar
\sphinxcode{\sphinxupquote{approx\_vf}}({[}n{]})
&
\sphinxAtStartPar
Method for approximating vf with power\sphinxhyphen{}law relation.
\\
\sphinxhline
\sphinxAtStartPar
\sphinxcode{\sphinxupquote{approx\_vf\_Kong}}({[}n{]})
&
\sphinxAtStartPar
Method for approximating vf from Kong.
\\
\sphinxhline
\sphinxAtStartPar
\sphinxcode{\sphinxupquote{area\_avg}}(param{[}, even\_opt, recalc{]})
&
\sphinxAtStartPar
Method for calculating the area\sphinxhyphen{}average of a parameter, "param"
\\
\sphinxhline
\sphinxAtStartPar
\sphinxcode{\sphinxupquote{calc\_W}}()
&
\sphinxAtStartPar
Calculates the wake deficit function, W, from the experimental data
\\
\sphinxhline
\sphinxAtStartPar
\sphinxcode{\sphinxupquote{calc\_avg\_lat\_sep}}()
&
\sphinxAtStartPar
Calculates average lateral separation distance between bubbles
\\
\sphinxhline
\sphinxAtStartPar
\sphinxcode{\sphinxupquote{calc\_cd}}({[}method, rho\_f, vr\_cheat, mu\_f{]})
&
\sphinxAtStartPar
Method for calculating drag coefficient
\\
\sphinxhline
\sphinxAtStartPar
\sphinxcode{\sphinxupquote{calc\_dpdz}}({[}method, rho\_f, rho\_g, mu\_f, ...{]})
&
\sphinxAtStartPar
Calculates the pressure gradient, dp/dz, according to various methods.
\\
\sphinxhline
\sphinxAtStartPar
\sphinxcode{\sphinxupquote{calc\_errors}}(param1, param2)
&
\sphinxAtStartPar
Calculates the errors, ε, between two parameters (param1 \sphinxhyphen{} param2) in midas\_dict
\\
\sphinxhline
\sphinxAtStartPar
\sphinxcode{\sphinxupquote{calc\_grad}}(param{[}, recalc{]})
&
\sphinxAtStartPar
Calculates gradient of param based on the data in self.
\\
\sphinxhline
\sphinxAtStartPar
\sphinxcode{\sphinxupquote{calc\_linear\_interp}}(param)
&
\sphinxAtStartPar
Makes a LinearNDInterpolator for the given param.
\\
\sphinxhline
\sphinxAtStartPar
\sphinxcode{\sphinxupquote{calc\_linear\_xy\_interp}}(param)
&
\sphinxAtStartPar
Makes a LinearNDInterpolator for the given param in x y coords
\\
\sphinxhline
\sphinxAtStartPar
\sphinxcode{\sphinxupquote{calc\_mu3\_alpha}}()
&
\sphinxAtStartPar
Calculates the third moment of alpha
\\
\sphinxhline
\sphinxAtStartPar
\sphinxcode{\sphinxupquote{calc\_mu\_eff}}({[}method, mu\_f, mu\_g, alpha\_max{]})
&
\sphinxAtStartPar
Method for calculating effective viscosity.
\\
\sphinxhline
\sphinxAtStartPar
\sphinxcode{\sphinxupquote{calc\_sigma\_alpha}}()
&
\sphinxAtStartPar
Calculates the second moment of alpha
\\
\sphinxhline
\sphinxAtStartPar
\sphinxcode{\sphinxupquote{calc\_vgj}}({[}warn\_approx{]})
&
\sphinxAtStartPar
Method for calculating Vgj, by doing
\\
\sphinxhline
\sphinxAtStartPar
\sphinxcode{\sphinxupquote{calc\_vgj\_model}}()
&
\sphinxAtStartPar
Method for calculating Vgj based on models
\\
\sphinxhline
\sphinxAtStartPar
\sphinxcode{\sphinxupquote{calc\_void\_cov}}()
&
\sphinxAtStartPar
Calculates the void covariance
\\
\sphinxhline
\sphinxAtStartPar
\sphinxcode{\sphinxupquote{calc\_vr}}({[}warn\_approx{]})
&
\sphinxAtStartPar
Method for calculating relative velocity.
\\
\sphinxhline
\sphinxAtStartPar
\sphinxcode{\sphinxupquote{calc\_vr\_model}}({[}method, c3, n, iterate\_cd, quiet{]})
&
\sphinxAtStartPar
Method for calculating relative velocity based on models
\\
\sphinxhline
\sphinxAtStartPar
\sphinxcode{\sphinxupquote{calc\_vwvg}}()
&
\sphinxAtStartPar
Calculates void weighted Vgj
\\
\sphinxhline
\sphinxAtStartPar
\sphinxcode{\sphinxupquote{circ\_segment\_area\_avg}}(param, hstar{[}, ...{]})
&
\sphinxAtStartPar
Method for calculating the area\sphinxhyphen{}average of a parameter, "param" over the circular segment defined by h
\\
\sphinxhline
\sphinxAtStartPar
\sphinxcode{\sphinxupquote{circ\_segment\_void\_area\_avg}}(param, hstar{[}, ...{]})
&
\sphinxAtStartPar
Method for calculating the void\sphinxhyphen{}weighted area\sphinxhyphen{}average of a parameter over the circular segment defined by h
\\
\sphinxhline
\sphinxAtStartPar
\sphinxcode{\sphinxupquote{find\_hstar\_pos}}({[}method, void\_criteria{]})
&
\sphinxAtStartPar
Returns the vertical distance from the top of the pipe to the bubble layer interface
\\
\sphinxhline
\sphinxAtStartPar
\sphinxcode{\sphinxupquote{fit\_spline}}(param)
&
\sphinxAtStartPar
Fits a RectBivariateSpline for the given param.
\\
\sphinxhline
\sphinxAtStartPar
\sphinxcode{\sphinxupquote{interp\_area\_avg}}(param{[}, interp\_type{]})
&
\sphinxAtStartPar
Function to area\sphinxhyphen{}average param, using the spline interpolation of param
\\
\sphinxhline
\sphinxAtStartPar
\sphinxcode{\sphinxupquote{line\_avg}}(param, phi\_angle{[}, even\_opt{]})
&
\sphinxAtStartPar
Line average of param over line defined by phi\_angle
\\
\sphinxhline
\sphinxAtStartPar
\sphinxcode{\sphinxupquote{line\_avg\_dev}}(param, phi\_angle{[}, even\_opt{]})
&
\sphinxAtStartPar
Second moment of param over line defined by phi\_angle
\\
\sphinxhline
\sphinxAtStartPar
\sphinxcode{\sphinxupquote{max}}(param{[}, recalc{]})
&
\sphinxAtStartPar
Return maximum value of param in the Condition
\\
\sphinxhline
\sphinxAtStartPar
\sphinxcode{\sphinxupquote{max\_line}}(param, angle)
&
\sphinxAtStartPar
Return maximum value of param at a given angle
\\
\sphinxhline
\sphinxAtStartPar
\sphinxcode{\sphinxupquote{max\_line\_loc}}(param, angle)
&
\sphinxAtStartPar
Return r/R location of maximum value of param at a given angle
\\
\sphinxhline
\sphinxAtStartPar
\sphinxcode{\sphinxupquote{max\_loc}}(param)
&
\sphinxAtStartPar
Return location of maximum param in the Condition
\\
\sphinxhline
\sphinxAtStartPar
\sphinxcode{\sphinxupquote{min}}(param{[}, recalc, nonzero{]})
&
\sphinxAtStartPar
Return minimum value of param in the Condition
\\
\sphinxhline
\sphinxAtStartPar
\sphinxcode{\sphinxupquote{min\_loc}}(param)
&
\sphinxAtStartPar
Return minimum value of param in the Condition
\\
\sphinxhline
\sphinxAtStartPar
\sphinxcode{\sphinxupquote{mirror}}({[}sym90, axisym, uniform\_rmesh, ...{]})
&
\sphinxAtStartPar
Mirrors data, so we have data for every angle
\\
\sphinxhline
\sphinxAtStartPar
\sphinxcode{\sphinxupquote{plot\_contour}}(param{[}, save\_dir, show, ...{]})
&
\sphinxAtStartPar
Method to plot contour of a given param
\\
\sphinxhline
\sphinxAtStartPar
\sphinxcode{\sphinxupquote{plot\_isoline}}(param, iso\_axis, iso\_val{[}, ...{]})
&
\sphinxAtStartPar
Plot profiles of param over iso\_axis at iso\_val
\\
\sphinxhline
\sphinxAtStartPar
\sphinxcode{\sphinxupquote{plot\_profiles}}(param{[}, save\_dir, show, ...{]})
&
\sphinxAtStartPar
Plot profiles of param over x\_axis, for const\_to\_plot, i.e. α over r/R for φ = {[}90, 67.5 .
\\
\sphinxhline
\sphinxAtStartPar
\sphinxcode{\sphinxupquote{plot\_spline\_contour}}(param{[}, save\_dir, show, ...{]})
&
\sphinxAtStartPar
Plots a contour from a spline interpolation
\\
\sphinxhline
\sphinxAtStartPar
\sphinxcode{\sphinxupquote{plot\_surface}}(param{[}, save\_dir, show, ...{]})
&
\sphinxAtStartPar
Method to plot a surface of a given param
\\
\sphinxhline
\sphinxAtStartPar
\sphinxcode{\sphinxupquote{pretty\_print}}({[}print\_to\_file, FID, mirror{]})
&
\sphinxAtStartPar
Prints out all the information in a Condition in a structured way
\\
\sphinxhline
\sphinxAtStartPar
\sphinxcode{\sphinxupquote{rough\_FR\_ID}}()
&
\sphinxAtStartPar
Identifies the flow regime for the given condition, by some rough methods
\\
\sphinxhline
\sphinxAtStartPar
\sphinxcode{\sphinxupquote{spline\_circ\_seg\_area\_avg}}(param, hstar{[}, int\_err{]})
&
\sphinxAtStartPar
Function to area\sphinxhyphen{}average over a circular segment defined by h, using the spline interpolation of param
\\
\sphinxhline
\sphinxAtStartPar
\sphinxcode{\sphinxupquote{spline\_void\_area\_avg}}(param)
&
\sphinxAtStartPar
Function to void\sphinxhyphen{}weighted area\sphinxhyphen{}average param over a circular segment defined by h, using the spline interpolation of param
\\
\sphinxhline
\sphinxAtStartPar
\sphinxcode{\sphinxupquote{top\_bottom}}(param{[}, even\_opt{]})
&
\sphinxAtStartPar
Honestly, I forgot what this does
\\
\sphinxhline
\sphinxAtStartPar
\sphinxcode{\sphinxupquote{void\_area\_avg}}(param{[}, even\_opt{]})
&
\sphinxAtStartPar
Method for calculating the void\sphinxhyphen{}weighted area\sphinxhyphen{}average of a parameter
\\
\sphinxbottomrule
\end{longtable}
\sphinxtableafterendhook
\sphinxatlongtableend
\end{savenotes}
\subsubsection*{Attributes}


\begin{savenotes}\sphinxattablestart
\sphinxthistablewithglobalstyle
\sphinxthistablewithnovlinesstyle
\centering
\begin{tabulary}{\linewidth}[t]{\X{1}{2}\X{1}{2}}
\sphinxtoprule
\sphinxtableatstartofbodyhook
\sphinxAtStartPar
\sphinxcode{\sphinxupquote{debugFID}}
&
\sphinxAtStartPar

\\
\sphinxbottomrule
\end{tabulary}
\sphinxtableafterendhook\par
\sphinxattableend\end{savenotes}

\end{fulllineitems}


\sphinxstepscope


\chapter{MARIGOLD.extracts\_and\_loads}
\label{\detokenize{modules/MARIGOLD.extracts_and_loads:module-MARIGOLD.extracts_and_loads}}\label{\detokenize{modules/MARIGOLD.extracts_and_loads:marigold-extracts-and-loads}}\label{\detokenize{modules/MARIGOLD.extracts_and_loads::doc}}\index{module@\spxentry{module}!MARIGOLD.extracts\_and\_loads@\spxentry{MARIGOLD.extracts\_and\_loads}}\index{MARIGOLD.extracts\_and\_loads@\spxentry{MARIGOLD.extracts\_and\_loads}!module@\spxentry{module}}\subsubsection*{Functions}


\begin{savenotes}\sphinxattablestart
\sphinxthistablewithglobalstyle
\sphinxthistablewithnovlinesstyle
\centering
\begin{tabulary}{\linewidth}[t]{\X{1}{2}\X{1}{2}}
\sphinxtoprule
\sphinxtableatstartofbodyhook
\sphinxAtStartPar
\sphinxcode{\sphinxupquote{dump\_data\_from\_tabs}}({[}dump\_file, skip\_dir{]})
&
\sphinxAtStartPar

\\
\sphinxhline
\sphinxAtStartPar
\sphinxcode{\sphinxupquote{extractIskandraniData}}({[}dump\_file{]})
&
\sphinxAtStartPar

\\
\sphinxhline
\sphinxAtStartPar
\sphinxcode{\sphinxupquote{extractLocalDataFromDir}}(path{[}, dump\_file, ...{]})
&
\sphinxAtStartPar
Function for getting all local data from spreadsheets in a directory, path
\\
\sphinxhline
\sphinxAtStartPar
\sphinxcode{\sphinxupquote{extractPitotData}}({[}dump\_file, in\_dir, ...{]})
&
\sphinxAtStartPar

\\
\sphinxhline
\sphinxAtStartPar
\sphinxcode{\sphinxupquote{extractProbeData}}({[}dump\_file, in\_dir, ...{]})
&
\sphinxAtStartPar

\\
\sphinxhline
\sphinxAtStartPar
\sphinxcode{\sphinxupquote{extractYangData}}({[}dump\_file{]})
&
\sphinxAtStartPar

\\
\sphinxhline
\sphinxAtStartPar
\sphinxcode{\sphinxupquote{loadData}}(data\_file)
&
\sphinxAtStartPar

\\
\sphinxhline
\sphinxAtStartPar
\sphinxcode{\sphinxupquote{loadIskandraniData}}({[}data\_file{]})
&
\sphinxAtStartPar

\\
\sphinxhline
\sphinxAtStartPar
\sphinxcode{\sphinxupquote{loadPitotData}}({[}data\_file{]})
&
\sphinxAtStartPar

\\
\sphinxhline
\sphinxAtStartPar
\sphinxcode{\sphinxupquote{loadProbeData}}({[}data\_file{]})
&
\sphinxAtStartPar

\\
\sphinxhline
\sphinxAtStartPar
\sphinxcode{\sphinxupquote{loadYangData}}({[}data\_file{]})
&
\sphinxAtStartPar

\\
\sphinxbottomrule
\end{tabulary}
\sphinxtableafterendhook\par
\sphinxattableend\end{savenotes}


\chapter{Indices and tables}
\label{\detokenize{index:indices-and-tables}}\begin{itemize}
\item {} 
\sphinxAtStartPar
\DUrole{xref,std,std-ref}{genindex}

\item {} 
\sphinxAtStartPar
\DUrole{xref,std,std-ref}{modindex}

\item {} 
\sphinxAtStartPar
\DUrole{xref,std,std-ref}{search}

\end{itemize}


\renewcommand{\indexname}{Python Module Index}
\begin{sphinxtheindex}
\let\bigletter\sphinxstyleindexlettergroup
\bigletter{m}
\item\relax\sphinxstyleindexentry{MARIGOLD.extracts\_and\_loads}\sphinxstyleindexpageref{modules/MARIGOLD.extracts_and_loads:\detokenize{module-MARIGOLD.extracts_and_loads}}
\end{sphinxtheindex}

\renewcommand{\indexname}{Index}
\printindex
\end{document}